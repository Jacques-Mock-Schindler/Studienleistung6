\documentclass{article}
\usepackage{amsmath}

\begin{document}

\title{Binäre Suche in mathematischer Notation}
\author{Autor Name}
\date{\today}

\maketitle

\section{Binäre Suche}

Die Funktion \textit{BinarySearch} wird definiert als:

\[
\text{BinarySearch}(A, x, \text{low}, \text{high}) = 
\begin{cases} 
-1 & \text{if } \text{high} < \text{low} \\
\text{mid} & \text{if } A[\text{mid}] = x \\
\text{BinarySearch}(A, x, \text{low}, \text{mid} - 1) & \text{if } x < A[\text{mid}] \\
\text{BinarySearch}(A, x, \text{mid} + 1, \text{high}) & \text{if } x > A[\text{mid}]
\end{cases}
\]

Hierbei ist \textit{A} ein sortiertes Array, \textit{x} der gesuchte Wert, \textit{low} der Anfangsindex und \textit{high} der Endindex des Suchbereichs. Der Index \textit{mid} wird als $\lfloor (\text{low} + \text{high}) / 2 \rfloor$ berechnet.

\end{document}
